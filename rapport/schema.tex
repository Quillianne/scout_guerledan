% Définition de la forme quelconque (un "blob") pour réutilisation.
% On définit des coordonnées polaires (angle:rayon) reliées par une courbe lisse.
\def\blobshape{ plot [smooth cycle, tension=0.8] coordinates {
    (0:1.5) (45:1.2) (90:1.6) (135:1.1) (180:1.4) (225:1.7) (270:1.0) (315:1.5)
} }

% ============================================================
% Fonctions utilitaires pour le pavage (SIVIA)
% ============================================================
\pgfmathdeclarefunction{normangle}{1}{\pgfmathparse{mod(#1,360) + (mod(#1,360) < 0 ? 360 : 0)}}
\pgfmathdeclarefunction{getradius}{1}{\pgfmathparse{
    (normangle(#1) < 45)  ? 1.5 + (normangle(#1)/45)*(1.2-1.5) : (
    (normangle(#1) < 90)  ? 1.2 + ((normangle(#1)-45)/45)*(1.6-1.2) : (
    (normangle(#1) < 135) ? 1.6 + ((normangle(#1)-90)/45)*(1.1-1.6) : (
    (normangle(#1) < 180) ? 1.1 + ((normangle(#1)-135)/45)*(1.4-1.1) : (
    (normangle(#1) < 225) ? 1.4 + ((normangle(#1)-180)/45)*(1.7-1.4) : (
    (normangle(#1) < 270) ? 1.7 + ((normangle(#1)-225)/45)*(1.0-1.7) : (
    (normangle(#1) < 315) ? 1.0 + ((normangle(#1)-270)/45)*(1.5-1.0) : 
                            1.5 + ((normangle(#1)-315)/45)*(1.5-1.5)
    ))))))))
}}
\pgfmathdeclarefunction{checkinside}{2}{\pgfmathparse{
    (sqrt(#1*#1+#2*#2) < getradius(atan2(#2,#1)) - 0.15) ? 1 :
    ((sqrt(#1*#1+#2*#2) > getradius(atan2(#2,#1)) + 0.15) ? -1 : 0)
}}

% ============================================================
% Pavage SIVIA (macro)
% ============================================================
\def\drawSiviaPavage#1#2{
    % #1 = couleur, #2 = opacité
    \foreach \x in {-2.4, -1.6, ..., 2.4} {
        \foreach \y in {-2.4, -1.6, ..., 2.4} {
            \pgfmathparse{int(checkinside(\x+0.4, \y+0.4))}
            \let\state\pgfmathresult
            \ifnum\state=1
                \draw[draw=#1!80!black, fill=#1!20, fill opacity=#2, thin] (\x,\y) rectangle ++(0.8,0.8);
            \else
                \foreach \xx in {0, 0.4} {
                    \foreach \yy in {0, 0.4} {
                        \pgfmathparse{int(checkinside(\x+\xx+0.2, \y+\yy+0.2))}
                        \let\stateMed\pgfmathresult
                        \ifnum\stateMed=1
                             \draw[draw=#1!80!black, fill=#1!20, fill opacity=#2, thin] (\x+\xx,\y+\yy) rectangle ++(0.4,0.4);
                        \fi
                        \ifnum\stateMed=0
                            \foreach \xxx in {0, 0.2} {
                                \foreach \yyy in {0, 0.2} {
                                     \pgfmathparse{int(checkinside(\x+\xx+\xxx+0.1, \y+\yy+\yyy+0.1))}
                                     \let\stateSmall\pgfmathresult
                                     \ifnum\stateSmall>-1
                                        \draw[draw=#1!80!black, fill=#1!20, fill opacity=#2, very thin] (\x+\xx+\xxx,\y+\yy+\yyy) rectangle ++(0.2,0.2);
                                     \fi
                                }
                            }
                        \fi
                    }
                }
            \fi
        }
    }
}


    % --- 2. Macro pour dessiner un bateau complet ---
    % Arguments : {Position}{Angle (0=Nord)}{Nom du noeud}{Couleur}{Label Bateau}{Label Vitesse}{Label Cap}{Label Box}
    \newcommand{\drawBoat}[8]{
        \begin{scope}[shift={(#1)}]
            % A. Bounding Box (Axis Aligned)
            % On dessine un rectangle centré sur la position
            \draw[bbox] (-1.0,-1.0) rectangle (1.0,1.0);
            \node[gray!60!black, anchor=north east] at (2.3,-0.8) {\small #8};

            % B. Axe de référence (Nord) pour l'angle theta
            \draw[dashed, thin, gray] (0,0) -- (0,1.5) coordinate (North);

            % C. Vecteur Vitesse
            % On calcule les coordonnées du vecteur vitesse
            \draw[velocity, color=#4!70!black] (0,0) -- (#2:1.8) node[anchor=south west] {#6};
            
            % D. Arc pour Theta (Cap)
            % On dessine l'arc entre le Nord et le vecteur vitesse
            \draw[->, thin, color=#4!80!black] (0,1) arc (90:#2:1) node[midway, anchor=south, font=\scriptsize] {#7};

            % E. Le Bateau lui-même
            % On le tourne selon l'angle
            \node[boat=#4, rotate=#2-90] (#3) at (0,0) {};
            
            % Label du bateau (au centre ou à côté)
            \node[font=\bfseries, color=#4!50!black, yshift=-1.5cm] at (0,0) {#5};
        \end{scope}
    }


% ============================================================
% Schéma 1 : prédiction simple (forme pleine)
% ============================================================
\newcommand{\SchemaPredictionSimple}{%
\begin{tikzpicture}[>=Stealth]

    % --- Configuration des positions ---
    % Centre de la première forme (en bas à droite)
    \coordinate (Pos1) at (0, 2);
    % Centre de la deuxième forme (en haut à gauche)
    \coordinate (Pos2) at (5, 5);

    % --- Dessin de la première forme C_{S1}(t-1) ---
    \begin{scope}[shift={(Pos1)}]
        % On dessine la forme avec un remplissage léger bleu
        \draw[thick, fill=blue!10, draw=blue!50!black] \blobshape;
        % Le label au centre
        \node at (0,0) {\large $C_{S1}(t-1)$};

        % On définit un point d'ancrage sur la frontière "haut-gauche" de cette forme pour le départ de la flèche.
        % (135 degrés est la direction haut-gauche, 1.2 est un rayon approximatif vers le bord)
        \coordinate (StartArrow) at (45:1.0);
    \end{scope}


    % --- Dessin de la deuxième forme C_{S1}(t) ---
    % On shift à la nouvelle position ET on applique une échelle (scale=1.4 pour agrandir)
    \begin{scope}[shift={(Pos2)}, scale=1.4]
        % On dessine la MÊME forme, avec un remplissage léger rouge
        \draw[thick, fill=red!10, draw=red!50!black] \blobshape;

        % Le label au centre. Note : comme on est dans un scope agrandi,
        % le texte serait aussi agrandi. On applique l'échelle inverse pour garder la taille du texte normale.
        \node[scale=1/1.4] at (0,0) {\large $C_{S1}(t)$};

        % On définit un point d'ancrage sur la frontière "bas-droite" pour l'arrivée de la flèche.
        % (-45 degrés est la direction bas-droite)
        \coordinate (EndArrow) at (-155:1.3);
    \end{scope}


    % --- La flèche ---
    % On relie les points d'ancrage définis dans les scopes
    \draw[->, ultra thick, color=black!70] (StartArrow) -- node[midway, above left, font=\bfseries] {d} (EndArrow);

\end{tikzpicture}
}

% ============================================================
% Schéma 2 : approche non récursive (pavage + translation)
% ============================================================
\newcommand{\SchemaPredictionNonRecursive}{%
\begin{tikzpicture}[>=Stealth]

    % --- Positions ---
    \coordinate (Pos1) at (0, 2);
    \coordinate (Pos2) at (6, 5);

    % --- FORME 1 : t-1 (forme + pavage transparent) ---
    \begin{scope}[shift={(Pos1)}]
        \draw[thick, blue!80!black] \blobshape;
        \drawSiviaPavage{blue}{0.3}
        \node[blue!90!black, fill=white, fill opacity=0.5, text opacity=1, rounded corners] at (0,0) {\large $C_{S1}(t-1)$};
        \node (LabelPave) at (0, -2.5) {\scriptsize $\text{pave}(C_{S1}(t-1), \varepsilon)$};
        \draw[->, thin, gray] (LabelPave) -- (0, -1.6);
        \coordinate (StartArrow) at (45:1.4);
    \end{scope}

    % --- FORME 2 : t (pavage opaque agrandi) ---
    \begin{scope}[shift={(Pos2)}, scale=1.4]
        \drawSiviaPavage{red}{0.9}
        \node[scale=1/1.4, fill=white, fill opacity=0.7, text opacity=1, rounded corners, inner sep=2pt] at (0,0) {\large $C_{S1}(t)$};
        \coordinate (EndArrow) at (-145:1.6);
    \end{scope}

    % --- Flèche d ---
    \draw[->, ultra thick, color=black!70] (StartArrow) -- node[pos=0.5, above left, font=\bfseries] {d} (EndArrow);

\end{tikzpicture}
}

% ============================================================
% Schéma 3 : intersection C_{S1}(t) ∩ C_{d01} ∩ C_{d12}
% (porté depuis test.tex)
% ============================================================
\newcommand{\SchemaIntersection}{%
\begin{tikzpicture}[>=Stealth]

    % --- Configuration des Centres et Rayons ---
    \coordinate (CenterBlue) at (-5., -5.);
    \coordinate (CenterGreen) at (3.5, -1.5);

    % Rayons (Largeur augmentée à 1.2)
    \def\RBlueMin{6.7}
    \def\RBlueMax{7.9}
    
    \def\RGreenMin{3.4}
    \def\RGreenMax{4.6}

    % --- 1. Dessin du Blob Rouge (Fond) ---
    \draw[thick, fill=red!5, draw=red!80!black] \blobshape;
    \node[red!80!black] at (-2.0, -1.4) {\Large $C_{S1}(t)$};

    % --- 2. Intersection TRIPLE (Logique Robuste) ---
    \begin{scope}[even odd rule]
        % A. Clip du blob
        \clip \blobshape;

        % B. Clip de la couronne Bleue (Deux cercles dans le même clip = anneau)
        \clip (CenterBlue) circle (\RBlueMax) (CenterBlue) circle (\RBlueMin);

        % C. Remplissage de la zone restante avec la couronne Verte
        \fill[red!60!black, opacity=0.8] 
            (CenterGreen) circle (\RGreenMax) (CenterGreen) circle (\RGreenMin);
    \end{scope}

    % --- 3. Dessin des contours (Arcs visibles) ---
    % Arcs Bleus (C_d01)
    \begin{scope}
        \clip (-2.5,-2.5) rectangle (2.5,2.5);
        \draw[very thick, blue] (CenterBlue) circle (\RBlueMin);
        \draw[very thick, blue] (CenterBlue) circle (\RBlueMax);
    \end{scope}
    \node[blue] at (-0.8, 2.1) {\large $C_{d_{01}}$};

    % Arcs Verts (C_d12)
    \begin{scope}
        \clip (-2.5,-2.5) rectangle (2.5,2.5);
        \draw[very thick, green!60!black] (CenterGreen) circle (\RGreenMin);
        \draw[very thick, green!60!black] (CenterGreen) circle (\RGreenMax);
    \end{scope}
    \node[green!60!black] at (0.9, 1.4) {\large $C_{d_{12}}$};

    % --- 4. Flèche et Label ---
    \node[right, draw=black!20, fill=white, fill opacity=0.9, rounded corners, inner sep=3pt] (LabelInter) at (3, 0) 
        {\small $C_{S1}(t) \cap C_{d_{01}} \cap C_{d_{12}}$};
    \draw[->, thick, black] (LabelInter.west) -- (0.2, 0.5);

\end{tikzpicture}
}

\newcommand{\SchemaModelisation}{%
\begin{tikzpicture}[
    >=Stealth, % Style de flèche
    % Style pour les bateaux (triangle simple)
    boat/.style={
        isosceles triangle,
        isosceles triangle apex angle=45,
        draw=##1!80!black,
        fill=##1!20,
        minimum height=1.2cm,
        shape border rotate=90, % Pointe vers le haut par défaut
        inner sep=0pt
    },
    % Style pour les bounding boxes
    bbox/.style={
        draw=gray!80,
        dashed,
        thick,
        fill=gray!5,
        fill opacity=0.5
    },
    % Style pour les vecteurs vitesse
    velocity/.style={
        ->,
        ultra thick,
        shorten <=2pt
    },
    % Style pour les mesures de distance
    measure/.style={
        <->,
        thick,
        color=blue!70!black
    }
]

    % --- 1. Définition des positions ---
    \coordinate (PosMS0) at (0, 0);     % Bateau mère en bas
    \coordinate (PosS1)  at (-4, 5);    % Scout 1 en haut à gauche
    \coordinate (PosS2)  at (4, 5.5);   % Scout 2 en haut à droite

    % --- 3. Dessin des trois bateaux ---

    % MS0 (Bateau Mère) - Cap légèrement vers la droite (80 degrés)
    \drawBoat{PosMS0}{120}{NodeMS0}{blue}{MS0}{$v_{MS0}$}{$\theta_{MS0}$}{$bb_{MS0}$}

    % S1 (Scout 1) - Cap vers le nord-ouest (110 degrés)
    \drawBoat{PosS1}{125}{NodeS1}{red}{S1}{$v_{S1}$}{$\theta_{S1}$}{$bb_{S1}$}

    % S2 (Scout 2) - Cap vers le nord-est (60 degrés)
    \drawBoat{PosS2}{115}{NodeS2}{green}{S2}{$v_{S2}$}{$\theta_{S2}$}{$bb_{S2}$}


    % --- 4. Dessin des Distances (Double flèches) ---
    
    % d01 : Entre MS0 et S1
    \draw[measure] (PosMS0) -- node[midway, left, font=\large, fill=white, inner sep=1pt] {$d_{01}$} (PosS1);

    % d20 : Entre MS0 et S2 (Note : d20 ou d02, c'est symétrique, je mets d20 comme demandé)
    \draw[measure] (PosMS0) -- node[midway, right, font=\large, fill=white, inner sep=1pt] {$d_{20}$} (PosS2);

    % d12 : Entre S1 et S2
    \draw[measure] (PosS1) -- node[midway, above, font=\large, fill=white, inner sep=1pt] {$d_{12}$} (PosS2);

\end{tikzpicture}
}
